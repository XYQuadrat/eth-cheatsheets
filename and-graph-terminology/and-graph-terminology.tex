\documentclass[a4paper,10pt]{article}

%----------------------------------------------------------------------------------------
%	FORMATTING
%----------------------------------------------------------------------------------------

\setlength{\parskip}{0pt}
\setlength{\parindent}{0pt}
\setlength{\voffset}{-15pt}

%----------------------------------------------------------------------------------------
%	PACKAGES AND OTHER DOCUMENT CONFIGURATIONS
%----------------------------------------------------------------------------------------

\usepackage[a4paper, margin=2.5cm]{geometry} % Sets margin to 2.5cm for A4 Paper
\usepackage[onehalfspacing]{setspace} % Sets Spacing to 1.5

\usepackage[T1]{fontenc} % Use European encoding
\usepackage[utf8]{inputenc} % Use UTF-8 encoding
\usepackage{charter} % Use the Charter font
\usepackage{microtype} % Slightly tweak font spacing for aesthetics

\usepackage[english, ngerman]{babel} % Language hyphenation and typographical rules

\usepackage{amsthm, mathtools, amssymb} % Mathematical typesetting
\usepackage{marvosym, wasysym} % More symbols
\usepackage{float} % Improved interface for floating objects
\usepackage[final, colorlinks = true, 
linkcolor = black, 
citecolor = black,
urlcolor = black]{hyperref} % For hyperlinks in the PDF
\usepackage{graphicx, multicol} % Enhanced support for graphics
\usepackage{xcolor} % Driver-independent color extensions
\usepackage{listings} % Environment for non-formatted code
\usepackage{booktabs} % Enhances quality of tables
\usepackage{tabularx}

\usepackage{titlesec} % Allows customization of titles
\titleformat{\section}{\large}{\currentSeries.\thesection}{1em}{}
\renewcommand\thesubsection{\alph{subsection})} % Alphabetic numerals for subsections
\titleformat{\subsection}[runin]{\large}{\thesubsection}{1em}{}
\renewcommand\thesubsubsection{\roman{subsubsection}.} % Roman numbering for subsubsections
\titleformat{\subsubsection}[runin]{\large}{\thesubsubsection}{1em}{}

% DATES
\usepackage[ddmmyyyy]{datetime}
\renewcommand{\dateseparator}{.}

% HEADER & FOOTER
\title{Graph Terminology Overview}
\author{Algorithms \& Datastructures}
\pagenumbering{gobble}

\begin{document}
  \maketitle
  \bigskip
  \begin{center}
    
    \renewcommand{\arraystretch}{1.3}
    \begin{tabularx}{\linewidth}{llr}
      \toprule
      walk & Weg & A series of connected vertices. \\
      trail & kantendisjunkter Weg & A walk without repeated edges. \\
      path & Pfad & A walk without repeated vertices. \\
      cycle\footnote{In some literature a cycle is also more generally equivalent to a circuit.} & Kreis & A path where \(v_0 = v_{end}\) holds.\footnote{This is not formally correct, since a path cannot have repeating vertices.} \\
      circuit, tour & kantendisjunkter Zyklus & A trail where \(v_0 = v_{end}\) holds. \\
      closed walk & Zyklus & A walk where \(v_0 = v_{end}\) holds.\\
      \midrule
      incident & inzident & connected (vertex \& edge)  \\
      adjacent & adjazent & neighboring (vertex \& vertex) \\
      reachable & \(u\) erreicht \(v\) & \(\exists\) walk from \(u\) to \(v\) \\
      connected & zusammenhängend & \(G\) has one connected component \\
      undirected & ungerichtet & all edges go both ways \\
      acyclic & azyklisch & no cycles in \(G\) \\
      \midrule
      degree & Grad & \# of edges incident to \(v\) \\
      indegree & Eingangsgrad & \# of incoming edges incident to \(v\) \\
      outdegree & Ausgangsgrad & \# of outgoing edges incident to \(v\) \\
      tree & Baum & connected graph without cycles \\
      leaf & Blatt & vertex with degree 1 \\
      forest & Wald & graph where every ZHK is a tree \\
      connected component & Zusammenhangskomponente & parts of a graph that are connected \\
      neighborhood & Nachbarschaft & subgraph of all vertices adjacent to \(v\)\\
      bridge, cut edge & Brücke & If \(e\) removed, \(G\) no longer connected \\
      articulation point, cut vertex & Artikulationsknoten& If \(v\) removed, \(G\) no longer connected \\
      \bottomrule
    \end{tabularx}
    \\ \bigskip 
  \end{center}
\end{document}